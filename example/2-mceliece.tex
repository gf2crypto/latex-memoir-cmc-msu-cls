%!TEX root = ../graduate-work.tex

\section{Криптосистема Мак-Элиса}
Опишем устройство криптосистемы с открытым ключом, предложенной Р.~
Дж.~Мак-Элисом~\cite{MCEliece}.

Пусть $\mathcal C$ --- некоторый линейный код с параметрами
$[n,k,d]$ над конечным полем $F_q$, который имеет эффективные
алгоритмы декодирования, $G$ --- его порождающая матрица размера
$k\times n$, $H$ --- невырожденная $k\times k$ матрица и $\Gamma$
--- перестановочная матрица размера $n\times n$. Секретным ключом
в данной схеме является тройка $(H,G,\Gamma)$, а открытым ключом
--- матрица $G'=H\cdot G\cdot \Gamma$. В оригинальной
криптосистеме Мак-Элиса~\cite{MCEliece} матрица $G$ выбирается
случайно из множества всех порождающих матриц некоторого класса
линейных кодов, например кодов Гоппы, с заданными параметрами.
Следует отметить, что иногда в секретный ключ не включается
матрица $G$, а она фиксируется и сообщается всем абонентам по
открытому каналу. Это связано с тем, что используемый класс
линейных кодов, например код Рида--Маллера RM(r,m), состоит из
одного единственного кода. Именно такая криптосистема
рассматривается в работах В.~М.~Сидельникова~\cite{Sidelnikov1} и
В.~М.~Сидельникова, С.~О.~Шестакова~\cite{Sidelnikov3}. Везде в
этой работе предполагается, что матрица $G$ --- фиксированная.

Опишем алгоритм зашифрования. На его вход подаётся открытое
сообщение $m$, которое является $k$-мерным вектором над полем
$F_q$. На выходе алгоритма образуется $n$-мерный вектор $c$,
который и является криптограммой открытого сообщения.

\textbf{Алгоритм Зашифрования.}
\begin{itemize}
\item[1.] Вычислить $c'=mH\cdot G\cdot \Gamma=mG'$.
\item[2.] Выбрать случайный $n$--мерный вектор $\mathbf e$ веса $wt(\mathbf e)=\lfloor\frac{d-1}2\rfloor$.
\item[3.] Вычислить $c=c'+\mathbf e$.
\end{itemize}

Опишем устройство алгоритма расшифрования. На его вход подаётся
криптограмма $c$
--- $n$-мерный вектор над полем $F_q$, а выходом является вектор $m$
--- зашифрованное сообщение.

\textbf{Алгоритм Расшифрования.}
\begin{itemize}
\item[1.] Вычислить $c'=c\Gamma^{-1}$;
\item[2.] Декодировать $c'$, то есть
представить его в виде $c'=aG+e'$, где $a=mH\in\mathcal
C,\;wt(e')=\lfloor\frac{d-1}2\rfloor;$
\item[3.] Вычислить секретное сообщение $m=aH^{-1}$.
\end{itemize}


Шаг 2 возможно выполнить в силу того, что перестановочная матрица
$\Gamma$ не изменяет веса вектора $e$, участвующего в алгоритме
шифрования, а только переставляет его координаты.

Злоумышленнику же  проделать шаги алгоритма расшифрования сложно,
так как он не знает матриц $H$ и $\Gamma$, поэтому ему трудно
декодировать код $\mathcal C$ с порождающей матрицей $G$, который
для него является кодом общего положения, а задача декодирования
такого кода является $\mathbf{\mathbb {NP}}$-трудной.

Известно~\cite{Evseev, Kruk}, что сложность $N$ декодирования
линейного кода общего положения имеет вид $N=2^{c\cdot
min(k,n-k)}.$ Откуда видно, что, даже при сравнительно небольших
параметрах $k\text{ и }n-k$, вычислительная сложность
декодирования таких кодов является неприемлемо высокой.

В оригинальной криптосистеме Мак-Элиса в качестве матрицы $G$ была
выбрана порождающая матрица двоичного кода Гоппы с параметрами
$[n=1024,k\ge 524,d\ge 101]$ над полем $\mathbb F_2$. Однако,
можно использовать порождающие матрицы любого другого кода,
имеющего быстрые и эффективные алгоритмы декодирования. Но не все
коды могут обеспечить необходимую стойкость криптосистемы
Мак-Элиса. Так, например, В.~М.~Сидельников и
С.~О.~Шестаков~\cite{Sidelnikov3} построили эффективный алгоритм
поиска секретного ключа по известному открытому в случае
использования порождающей матрицы обобщённого кода Рида--Соломона.
