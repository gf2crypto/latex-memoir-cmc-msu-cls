%!TEX root = ../graduate-work.tex

\section{Криптосистема Мак-Элиса--Сидельникова}
В.~М.~Сидельников в работе~\cite{Sidelnikov1} провёл
криптографический анализ криптосистемы Мак-Элиса, в которой в
качестве кода $\mathcal{C}$ выбирался двоичный код Рида--Маллера
$RM(r,m)$ с порождающей матрицей $R$. В результате криптоанализа
В.~М.~Сидельников пришёл к выводу, что на сегодняшний день данная
криптосистема не обеспечивает необходимого уровня стойкости. В
этой же работе автор предложил некоторую её модификацию. Опишем
конструкцию, предложенную В.~М.~Сидельниковым.

\Def{\emph{Кодом Рида--Маллера} $RM(r,m)$, называется множество
векторов значений $\Omega_f$ всех булевых функций
$f(y_1,\ldots,y_m)$, степень нелинейности (максимальная длина
монома, входящего в полином Жегалкина функции $f$) которых не
превосходит $r$, то есть
\begin{eqnarray*}
RM(r,m)&=&\{\Omega_f=(x_1,\ldots,x_n),
n=2^m|\\&&f(y_1,\ldots,y_m)=a_0\oplus\bigoplus_{s=1}^{t}\bigoplus_{1\leqslant
i_1<\ldots<i_s\leqslant m}a_{i_1,\ldots,i_s}y_{i_1}\ldots
y_{i_s},\;t\leqslant r\}
\end{eqnarray*}
} Код $RM(r,m)$ имеет размерность $k=\sum_{i=0}^{r}\binom{m}{i}$,
длину $n=2^m$ и кодовое расстояние $d=2^{m-r}$,
см.~\cite{McWilliams}. Обозначим через $R$ порождающую матрицу
кода Рида--Маллера $RM(r,m)$, которая состоит из единичного
вектора и векторов-значений всех мономов от $m$ переменных степени
нелинейности не превосходящей $r$.
$$
R=\begin{pmatrix} G_0\\
G_1\\
\vdots\\
G_r
\end{pmatrix},
$$
где $G_0=(1,1\ldots,1),$
$$
G_1=\begin{pmatrix}\Omega_{y_m}\\
\vdots\\
\Omega_{y_2}\\ \Omega_{y_1}
\end{pmatrix},
G_2=\begin{pmatrix}\Omega_{y_{m-1}y_m}\\
\vdots\\
\Omega_{y_1y_3}\\
\Omega_{y_1y_2}
\end{pmatrix},
G_r=\begin{pmatrix}\Omega_{y_{m-r+1}y_{m-r+2}\ldots y_m}\\
\vdots\\
\Omega_{y_1y_2\ldots y_{r-1}y_{r+1}}\\
\Omega_{y_1y_2\ldots y_{r-1}y_r}
\end{pmatrix}
$$
Матрицу $R$ ещё будем называть \emph{стандартной формой}
порождающей матрицы кода Рида--Маллера $RM(r,m)$.

Секретным ключом криптосистемы уже являются не две матрицы, а
кортеж
$$(H_1,H_2,\ldots,H_u,\Gamma).$$ Здесь $H_1,H_2,\ldots,H_u$
--- невырожденные $k\times k$-матрицы над полем $F_2=\{0,1\}$,
которые выбираются случайно и равновероятно из множества всех
двоичных невырожденных $k\times k$-матриц. Матрица $\Gamma$ имеет
размеры $u\cdot n\times u\cdot n$ и является перестановочной, то
есть в каждой её строчке и в каждом столбце стоит ровно одна
единица. Заметим, что в секретный ключ не включается матрица $R$,
так как это не имеет смысла в силу единственности кода
Рида--Маллера $RM(r,m)$.

Открытым ключом криптосистемы Мак-Элиса--Сидельникова является
матрица
$$G'=(H_1R\|H_2R\|\ldots\|H_uR)\cdot \Gamma,$$
где символом $\|$ обозначена конкатенация матриц. Алгоритм
зашифрования в такой криптосистеме почти не отличается от
классического. Для зашифрования секретного сообщения $m$, длины
$k$ нужно

\textbf{Алгоритм Зашифрования.}
\begin{itemize}
\item[1.] Вычислить $c'=mG'$.
\item[2.] Выбрать случайный $(u\cdot n)$--мерный
вектор $\mathbf e$ такой, что для его веса выполняется $wt(\mathbf
e)\le u\cdot\lfloor\frac{d-1}2\rfloor+u-1$.
\item[3.] Вычислить $c=c'+\mathbf e$.
\end{itemize}


Следует заметить, что длина криптограммы $c$ равна $un$. Из
алгоритма зашифрования видно, что в криптосистеме
Мак-Элиса--Сидельникова каждое открытое сообщение имеет
большее число возможных криптограмм, чем в оригинальной
криптосистеме. Опишем теперь алгоритм расшифрования криптограммы
$c$.

\textbf{Алгоритм Расшифрования.}
\begin{itemize}
\item[1.] Вычислить $c'=c\Gamma^{-1}$.
\item[2.] Представить вектор $c'$ в виде $c'=(c'_1\|\ldots\|c'_u)$,
где $c'_i\in F^n_2$.
\item[3.] Каждый вектор $c'_i$ попытаться представить в виде $c'_i=a_iR+e'_i$,
для некоторого $a_i\in F^k_2$  и некоторого вектора ошибок
$e'_i\in F^n_2$ веса, не превосходящего
$\lfloor\frac{d-1}2\rfloor$.
\item[4.] Взять любое $c'_i$, которое удалось представить в виде,
указанном в пункте 3.
\item[5.] Вычислить $m=a_iH^{-1}_i$.
\end{itemize}


Заметим, что алгоритм расшифрования корректен. Действительно,
вектор $c'$ отстоит от кода с порождающей
матрицей $(H_1R\|\ldots\|H_uR)$ на расстояние не превосходящее
$u\cdot\lfloor\frac{d-1}2\rfloor+u-1$. Следовательно, найдётся
вектор $c'_i$, отличающийся от некоторого вектора кода $RM(r,m)$
не более, чем в $\lfloor\frac{d-1}2\rfloor$ позициях. Именно для
этого вектора и удастся получить разложение из пункта 3 алгоритма
зашифрования.

Для кодов Рида--Маллера существуют эффективные алгоритмы
декодирования в пределах кодового расстояния~\cite{McWilliams}, а
это обуславливает высокую скорость расшифрования. Заметим также,
что существуют хорошие алгоритмы декодирования кодов Рида--Маллера
и при числе ошибок превосходящем половину кодового
расстояния~\cite{Sidelnikov2}. Очевидный недостаток
--- низкая скорость передачи, которая здесь хуже, чем в оригинальной схеме на
основе тех же самых кодов.

Одно из самых первых направлений исследования новой криптосистемы
--- изучение множества открытых ключей.
Обозначим через $\mathcal E$ множество всех открытых ключей
криптосистемы Мак-Элиса--Сидельникова. Возникает вопрос о мощности
множества $\mathcal E$. В.~М.~Сидельников в своей
работе~\cite{Sidelnikov1} предложил следующую гипотезу
$$|\mathcal E|=\frac{(un)!(h_k)^u}{u!|Aut(RM(r,m))|^u},$$
где $h_k=(2^k-1)(2^k-2)\ldots (2^k-2^{k-1})$ --- число
невырожденных матриц размерности $k$ над полем $F_2$;
$|Aut(RM(r,m))|=2^m(2^m-1)\ldots(2^m-2^{m-1})$
--- мощность группы автоморфизмов кода Рида-Маллера $RM(r,m)$.

Эта гипотеза оказалась ошибочной. Г.~А.~Карпунин
доказал~\cite{Karpunin}, что
$$|\mathcal E|<\frac{(un)!(h_k)^u}{u!|Aut(RM(r,m))|^u}.$$
Он также полностью описал множество открытых ключей $\mathcal E$
для случая кода $RM(1,m)$ при $u=2$ и вычислил его мощность.
